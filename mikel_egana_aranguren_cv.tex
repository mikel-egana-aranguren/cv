
%% http://theprofessorisin.com/2012/01/12/dr-karens-rules-of-the-academic-cv/

%% forked from https://github.com/cmungall/cv

\documentclass[11pt,fullpage]{article}

\usepackage{hyperref}
\usepackage{geometry}
%\usepackage{listliketab}
\usepackage{array}
\usepackage{longtable}

\usepackage{natbib}
\usepackage{bibentry}
\nobibliography*
%\bibliographystyle{apa}

% Palatino font
%\usepackage[T1]{fontenc}
%\usepackage[sc,osf]{mathpazo}


\usepackage{xspace}
\usepackage{color}

\newcommand{\myurl}[1]{\footnote{\url{#1}}}
\newcommand{\via}{\emph{via}\xspace}
\newcommand{\ie}{\emph{i.e.}\xspace}
\newcommand{\eg}{\emph{e.g.}\xspace}
\newcommand{\etc}{\emph{etc.}}
\newcommand{\todo}[1]{\textbf{{\color{blue}$\Longrightarrow$ #1}}}

\def\name{Mikel Ega\~na Aranguren, Ph.D.}

\reversemarginpar

\geometry{
  body={6.5in, 8.5in},
  left=1.0in,
  top=1.0in
}

% Customize page headers
\pagestyle{myheadings}
\markright{\name}
\thispagestyle{empty}

% Custom section fonts
%\usepackage{sectsty}
%\sectionfont{\rmfamily\mdseries\Large}
%\subsectionfont{\rmfamily\mdseries\itshape\large}

% Other possible font commands include:
% \ttfamily for teletype,
% \sffamily for sans serif,
% \bfseries for bold,
% \scshape for small caps,
% \normalsize, \large, \Large, \LARGE sizes.

% Don't indent paragraphs.
\setlength\parindent{0em}

% Make lists without bullets
\renewenvironment{itemize}{
  \begin{list}{}{
    \setlength{\leftmargin}{1.5em}
  }
}{
  \end{list}
}

\begin{document}
\bibliographystyle{apalike}

% Print name centered and bold:
\centerline{\Large \bf \name}

\vspace{0.25in}

\begin{minipage}{0.50\linewidth}
~\\
    Assistant Profesor \\




%   \href{http://www.genomic-resources.eu/}{Genomic Resources Group} \\
  Office P3I16 \\ 
  Department of Computer Languages and Systems \\
  School of Engineering \\
  University of Basque Country (UPV/EHU) \\
  48013 Bilbao, Spain \\
  % Calle Rafael Moreno "Pitxitxi", 3 | 48013 Bilbao, Spain \\


\end{minipage}
\begin{minipage}{0.50\linewidth}
  \begin{tabular}{ll}
%     Tel: & (0034) 946015503 \\
%    Skype: & mikel\_egana \\
%     Cell: & (0034) 656724785 \\
    @: 
& \href{mailto:mikel.egana@ehu.eus}{mikel.egana@ehu.eus} \\
    www: & \href{https://mikel-egana-aranguren.github.io}{https://mikel-egana-aranguren.github.io} \\
    Phone: & +34 94 601 4464 \\

%     (Up-to-date CV, pointers to profiles (CiteUlike, GitHub, LinkedIn, \etc), links to publications, and other works (slides, theses, teaching materials, videos, \etc)) \\
  \end{tabular}
\end{minipage}

% \section*{Short Bio}

% I am a seasoned Linked Data expert, with hands-on experience in the design and development of data integration solutions, from ontology engineering to mapping of data sources to Knowledge Graphs. I hold an academic position but I have also worked in the private sector, in different companies (Notably Eccenca GmbH) and as a freelance.

% My research interests revolve around the idea of making the publication of data following FAIR principles (Findable, Accessible, Interoperable, Reusable) easier and more efficient.

\section*{Education}

\begin{tabular}{ll}
	2009 & {\bf Ph.D.} Computer Science, University of Manchester, UK \\
	2005 & {\bf M.Sc.} Bioinformatics, University of Manchester, UK \\
	2003 & {\bf B.Sc.} Biology, University of Basque Country, Spain \\
	2002 & {\bf Invited student} Evolutionary Ecology at Canterbury Christ Church University \\
	     & College, UK \\
	2002 & {\bf Erasmus student} Environmental Biology at Canterbury Christ Church University \\
	     & College, UK \\
\end{tabular}

\section*{Employment}

% Give institution, department, title, and dates (year only) of
% employment.  Be sure and reflect joint appointments if you have one

\begin{tabular}{ll}
2020/09/01 - Present & {\bf Assistant Profesor}, Dept. of Computer Languages and Systems, \\
                    & UPV/EHU \\
2018/12/02 - 2020/08/31 & {\bf Linked Data Consultant} Eccenca GmbH \\
      & Development of Linked Data solutions for the Enterprise, \\ 
      & including project management\\
2018/10/01 - 2018/12/02 & {\bf Bioinformatics technician} Biocruces Bizkaia \\
      & Data infrastructure development, Bioinformatics analyses \\
2016/01/14 - 2018/09/30 & {\bf Analyst, Torres Quevedo fellow} Eurohelp Consulting \\
	      & Design and development of Linked Open Data solutions\\
 2015/04/27 -  2016/01/14 & {\bf Analyst} Eurohelp Consulting \\
	    & Design and development of Linked Open Data solutions\\
\end{tabular}

\section*{Research positions}
\begin{tabular}{ll}
2014/04/01 - 2015/03/31 & {\bf Post-doc researcher (80\% FTE)} Genomic Resources Group, UPV/EHU  \\
   & Metagenomics and Life Sciences Semantic Web \\
2011/02/14 - 2014/02/14 & {\bf Post-doc researcher, Marie Curie Cofund fellow} \\
     & Ontology Engineering Group (Computer Science); \\
     & Biological Informatics Group (CBGP), UPM, Spain \\
     & Ontology Engineering and Life Sciences Semantic Web \\
2010/12/01 - 2011/02/01 & {\bf Researcher} OGO project, UM, Spain \\
     & Orthologous Genes Ontology \\
2006/05/01 - 2006/10/1 & {\bf Pre-doc researcher, Marie Curie EST fellow} \\
    & Computational Biology group, VIB, Belgium \\
    & Cell Cycle Ontology and Ontology Design Patterns \\
\end{tabular}

\section*{Freelance positions}
\begin{tabular}{ll}

2018/10/15 - 2018/12/02 & {\bf Ontology Engineering}. Cognizone, Belgium\\
2018/05/01 - 2018/07/31 & {\bf Public tender technical writer}. University of Murcia \\
        &  Project Hercules: Federated Linked Open Data for universities\\
2016/10/04 - 2017/02/25 & {\bf Ontology Engineering}. Intellimedis, Luxembourg\\
\end{tabular}

\section*{Research visits}

\begin{tabular}{ll}
 2005/09/01 - 2005/10/01 & European Bioinformatics Institute (EBI), funded by the Network of Excellence \\
      & on Semantic Interoperability and Data Mining in Biomedicine (EU)

\end{tabular}

\section*{Funding obtained}

\begin{tabular}{ll}
%\begin{longtable}{p{0.5in}|p{5.5in}}
  2018-2020 (2 years) & {\bf Declined} Bioinformatics Technician. Instituto de Salud Carlos III. CA18/00021 \\
  2016-2018 (3 years) & Torres Quevedo (Spain). 35\% of salary at Eurohelp Consulting. PTQ-14-07198 \\
	2011-2014 (3 years) & Marie Curie Cofund (EU). UNITE 246565 \\
	2006 (5 months) & Marie Curie EST (EU). MEST-CT-2004-414632 \\
	2005 (One payment) & EPSRC (UK): Ph.D. fees \\
	2005 - 2008 (3 years) & University of Manchester (UK): Ph.D. maintenance allowance \\
	2002 (5 months) & Erasmus (EU) \\
%\end{longtable}
\end{tabular}

\section*{Participation in projects}

\subsection*{Research}

\begin{tabular}{ll}
    2021 - 2024 & SUPPORT4LS (Process Mining and Knowledge Representation technologies to \\ 
                & Support the Learning Health System). PID2020-113723RB-C22. \\ 
                & Agencia Estatal de Investigación, ``Proyectos I+D+i 2020'' - \\ 
                & Modalidades ``Retos Investigación'' y ``Generación de Conocimiento'' \\ 
                & (Plan Estatal de Investigación Científica y Técnica y de Innovación 2017-2020). \\
                & 136.972,00 EUR. 01/09/2021 - 31/08/2024 \\
    2021 - 2022 &  Producci\'on de Datos Enlazados para Open Data Euskadi. EJIE (Sociedad Inform\'atica \\ 
                &  del Gobierno Vasco), 2105015. Production of Linked Data for Open Data Euskadi \\ 
                & (9.000 EUR) Knowledge transfer project (TR41652) managed by \href{http://www.euskoiker.ehu.es/faq-3}{Euskoiker foundation}. \\ 
    2017 - 2018 & SOLDAGE (Semantic Open Linked DAta GEnerator). HAZITEK, Gobierno Vasco. \\
                & FAIR data generator (150.000 EUR). \\
    2016 - 2020 & REPLICATE. Renaissance of Places with Innovative Citizenship And \\
                & TEchnology (Project 691735), EU. Linked Open Data in Smart Cities \\
                & (\href{http://replicate-project.eu/}{http://replicate-project.eu/}). \\
                & (Consortium: 24.965.263,09; Eurohelp: 328.580,00 EUR).  \\
    2015 - 2017 & Linking Open Domains, Plataforma para la generaci\'on de datos enlazados  \\
                & (LODGen) (TSI-100105-2015-0012). Ministerio de Industria, Energia y Turismo (Spain), \\
                & Acción Estratégica Economía y Sociedad Digital (AEESD) 1/2015.  \\
                & Linked Open Data pipeline (40.182,54 EUR). \\
    2015 - 2016 & Enlazando Gipuzkoa con el Mundo (ENGIMU). Gipuzkoako Foru Aldundia,  \\
         & Gipuzkoa IKT: Innovaci\'on digital Empresas (Spain).\\
         & Linked Open Data pipeline (40.000 EUR). \\
  \end{tabular}

\subsection*{Development}

\begin{tabular}{ll}

2018 & \href{http://www.ejie.eus/y79-contgen/es/contenidos/anuncio_contratacion/expx74j30109/es_doc/es_arch_expx74j30109.html?ruta=/y79-appcontr/es/v79aWar/comunJSP/v79aSuscribirRSS.do?=R01HPortal=y79&R01HPage=appcontr&R01HLang=es&widget=true&p01=AC&p02=&p03=8&p04=&p05=&p06=&p07=&p08=&p09=&p10=&p11=&p12=&p13=&p14=&p15=05%2F04%2F2018&p16=&p17=AMPLIADO&p18=false&p19=false&p20=false&p21=es&p22=ultimos30dias&p23=&p24=y79-appcontr&p25=y79-contgen&p45=true&p48=&p51=1}{Servicios Directorio Linked Open Data}. \\
& EJIE (Sociedad Inform\'atica del Gobierno Vasco). (70.000 EUR). \\
2016 - 2017 & \href{http://www.contratacion.euskadi.eus/w32-1084/es/contenidos/anuncio_contratacion/expx74j21656/es_doc/es_arch_expx74j21656.html}{Servicios Open Linked Data}. EJIE (Sociedad Inform\'atica del Gobierno Vasco). \\
	          & Linked Data implementation of Open Data Euskadi (90.000 EUR). \\
\end{tabular}

\section*{Publications}

\subsection*{Refereed Journal Articles}

\setlength{\extrarowheight}{10pt}

\begin{longtable}{p{0.5in}|p{5.5in}}
% Use: ./util/parsebib.pl article MyPubs.bib  | sort -r

 2015 & \bibentry{aranguren2015-gigascience} \\
 2015 & \bibentry{Pawluczyk-ABC} \\
 2014 & \bibentry{AleSr2014JBMS-OpenLifeData-SADI} \\
 2014 & \bibentry{aranguren2014JBMS-SADI-Galaxy} \\
 2014 & \bibentry{aranguren2014SWJ-ogolod} \\
 2014 & \bibentry{aranguren2014SWJ} \\
 2013 & \bibentry{oquare2013} \\
 2013 & \bibentry{EganaAranguren2013} \\
 2012 & \bibentry{minarro2012publishing} \\
 2011 & \bibentry{mironov2011flexibility} \\
 2011 & \bibentry{micnarro2011semantic} \\
 2009 & \bibentry{antezana2009cell} \\
 2009 & \bibentry{antezana2009biogateway} \\
 2008 & \bibentry{egana2008situ} \\
 2008 & \bibentry{aranguren2008ontology} \\
 2008 & \bibentry{antezana2008onto} \\
 2007 & \bibentry{stevens2007using} \\
 2007 & \bibentry{aranguren2007understanding} \\
\end{longtable}

\subsection*{Book Chapters}

\begin{longtable}{p{0.5in}|p{5.5in}}
% Use: ./util/parsebib.pl incollection MyPubs.bib  | sort -r
2010 & \bibentry{aranguren2010technologies} \\
\end{longtable}

\subsection*{Books}

\begin{longtable}{p{0.5in}|p{5.5in}}
% Use: ./util/parsebib.pl book MyPubs.bib  | sort -r
2010 & \bibentry{phd_mikel} \\
\end{longtable}


\subsection*{Conference Proceedings}
% Use: ./util/parsebib.pl inproceedings MyPubs.bib  | sort -r

\begin{longtable}{p{0.5in}|p{5.5in}}

 2014 & \bibentry{alesr2014} \\
 2013 & \bibentry{iwbbio2013} \\
 2011 & \bibentry{aranguren2011oppl} \\
 2010 & \bibentry{minarro2010semantic} \\
 2008 & \bibentry{ekaw2008} \\

\end{longtable}

\subsection*{Preprints, Workshop Proceedings and other publications}
% Use: ./util/parsebib.pl inproceedings MyPubs.bib  | sort -r
% Watch out for "OTHER-"

% Use: ./util/parsebib.pl misc MyPubs.bib  | sort -r

\begin{longtable}{p{0.5in}|p{5.5in}}
 2015 & \bibentry{OTHER-bioRxiv-SADI-Galaxy-Docker} \\
 2012 & \bibentry{OTHER-horridge2012ontology} \\
 2012 & \bibentry{OTHER-gimenez2012ncbo} \\
 2009 & \bibentry{OTHER-fernandez2009quality} \\
 2009 & \bibentry{OTHER-aranguren2009transforming} \\
 2008 & \bibentry{OTHER-iannone2008augmenting} \\
 2008 & \bibentry{OTHER-antezana2008structuring} \\
 2012 & \bibentry{OTHER-marshall2012w3c} \\
 2007 & \bibentry{OTHER-biogaia7}\\
 2003 & \bibentry{OTHER-biogaia3}
\end{longtable}

\section*{Committees and reviewer work}

% Include journal manuscript review work (with journal titles
% [mss. review CAN be given its own separate heading if you do a lot
% of this work]), leadership of professional organizations, etc. Some
% people put panel organizing under service; check conventions in your
% field.

\begin{longtable}{p{0.5in}|p{5.5in}}

2022 & \textbf{Program Committee Member} at Semantic Web Solutions for Large-Scale Biomedical Data Analytics (SeWeBMeDA 2022) \\
2020 & \textbf{Program Committee Member} at Semantic Web Solutions for Large-Scale Biomedical Data Analytics (SeWeBMeDA 2020) \\
2019 & \textbf{Program Committee Member} at Semantic Web Solutions for Large-Scale Biomedical Data Analytics (SeWeBMeDA 2019) \\  
2017 & \textbf{Chapter review} ``Integrating Biological Data using Semantic Web Technology'' in ``Evolutionary Genomics. Computational and statistical methods'', 3rd edition, Springer. \\
2017 & \textbf{Program Committee Member} at Semantic Web Solutions for Large-Scale Biomedical Data Analytics (SeWeBMeDA 2017) \\
2015 & \textbf{Program Committee Member} at Linked Data workshop (CAEPIA 2015) \\
2015  & \textbf{Reviewer} for BMC Medical Informatics and Decision Making \\
2013 & {\bf Special issue editor} for Semantic Web Journal (SWJ): Special issue on Linked Data for Health Care and the Life Sciences \\
2013  & \textbf{Reviewer} for PeerJ \\
2013  & \textbf{Reviewer} for Data and Knowledge Engineering (DKE) \\
2012 & \textbf{Program Committee Member} at Managing Interoperability and compleXity in Health Systems. In conjunction with the ACM International Conference on Information and Knowledge Management\\
2012 & \textbf{Program Committee Member} at Joint Workshop on Semantic Technologies Applied to Biomedical Informatics and Individualized Medicine (SATBI + SWIM 2012). In conjunction with International Semantic Web Conference (ISWC)\\
2012  & \textbf{Reviewer} for BMC Bioinformatics \\
2012  & \textbf{Reviewer} for Journal of Biomedical Informatics (JBI) \\
2012  & \textbf{Reviewer} for Computational and Mathematical Methods in Medicine (CMMM) \\
2012  & \textbf{Reviewer} for Journal of Medical Systems (JOMS) \\
2012  & \textbf{Reviewer} for Journal of Biomedical Semantics (JBS) \\
2011 & \textbf{Program Committee Member} at Managing Interoperability and compleXity in Health Systems. In conjunction with the ACM International Conference on Information and Knowledge Management\\
2011 & \textbf{Program Committee Member} at Knowledge Capture (K-CAP)\\
2011 & \textbf{Program Committee Member} at Semantic Applied Technologies on Biomedical Informatics (SATBI 2011). In conjunction with the ACM International Conference on Bioinformatics and Computational Biology\\
2011  & \textbf{Reviewer} for Semantic Web Journal (SWJ) \\
2011  & \textbf{Reviewer} for Journal of Research and Practice in Information Technology (JRPIT)\\
2008 & \textbf{Program Committee Member} at ONTORACT \\

\end{longtable}



%\section*{Additional Papers Presented at Professional Meetings}

\section*{Invited Talks}

\begin{longtable}{p{0.5in}|p{5.5in}}

% Give title, institutional location, and date. Year only (not month
% or day) at left.  Month and day of talk go into entries.

2016 & Los Datos Enlazados y la Web Sem\'antica. Tikitalka, VE Interactive, Spain \\

2014 & Building reasonable biomedical ontologies for a Life Sciences Semantic Web. 3S (Systems, Synthetic, and Semantic) Biology summer school. CIBIO (Centre for Integrative Biology), University of Trento, Italy \\

2011 & Linked Data for Functional Genomics. NTNU, Trondheim, Norway \\

2010 & Aplicaci\'on de la Web Sem\'antica en Biolog\'ia Molecular. Universidad de Deusto, Facultad de Ingenier\'ia, Spain \\

2008 & Aplicaci\'on de la Web Sem\'antica en Bioinform\'atica. UM, Facultad de Inform\'atica, Spain \\

2004 & M\'etodos y resultados actuales en Bioinform\'atica: know-how y know-what de las redes tecnocient\'ificas en Bioinform\'atica. EHU, Facultad de Filosof\'ia, Spain

\end{longtable}


%\section*{Conference Activity}
% see below

% These entries will include: Name of paper, name of conference,
% date. Year (Year only) on left as noted above. Month and date-range
% of conference in the entry itself (ie, March 22-25).

\section*{Teaching Experience}

\begin{longtable}{p{0.5in}|p{5.5in}}
2021 & FAIR data. MSc Bioinformatics, UM. Spanish \\
2021 & Information Security (6 ECTS). Degree in Computer Engineering of Management and Information Systems, UPV/EHU. Spanish \\
2021 & Analysis and Design of Information Systems (6 ECTS). Degree in Computer Engineering of Management and Information Systems, UPV/EHU. Spanish \\
2021 & Project Management (3 ECTS). Degree in Computer Engineering of Management and Information Systems, UPV/EHU. Spanish and Basque \\
2020 & FAIR data. MSc Bioinformatics, UM. Spanish \\
2020 & Information Security (6 ECTS). Degree in Computer Engineering of Management and Information Systems, UPV/EHU. Spanish \\
2020 & Analysis and Design of Information Systems (6 ECTS). Degree in Computer Engineering of Management and Information Systems, UPV/EHU. Spanish \\
2017 & Life Sciences Semantic Web. MSc Bioinformatics, UM. Spanish \\
2016 & Life Sciences Semantic Web. MSc Bioinformatics, UM. Spanish \\
2016 & Linked Open Data tutorial. EJIE. Spanish \\
2015 & Linked Open Data tutorial. IZFE (Informatika Zerbitzuen Foru Elkartea, Gipuzkoa). Spanish \\
2015 & Life Sciences Semantic Web. MSc Bioinformatics, UM. Spanish \\
2014 & Semantic biology tutorial: Use of Semantic Web resources for knowledge discovery. 3S (Systems, Synthetic, and Semantic) Biology summer school. CIBIO (Centre for Integrative Biology), University of Trento, Italy. English \\
2014 & Galaxy tutorial. Erasmus mundus MSc in Marine Environment and resources, UPV-EHU. English \\
2014 & Life Sciences Semantic Web. MSc Bioinformatics, UM. Spanish \\ %[\href{https://github.com/mikel-egana-aranguren/MSc_Bioinformatics_UM_13-14_LSSW}{Material}]
2013 & Introductory talk on bioinformatics for high school students visiting the CBGP. Spanish \\
2013 & Galaxy tutorials at CBGP. English and spanish \\ %[\href{http://github.com/mikel-egana-aranguren/Galaxy\_tutorial}{http://github.com/mikel-egana-aranguren/Galaxy\_tutorial}]
2013 & Life Sciences Linked Data. MSc Bioinformatics, UM. Spanish \\
2012 & OWL, as part of ATHENS course (UPM). English \\
2012 & OWL, as part of ATHENS course (UPM). English \\
2011 & Populous tutorial at SWAT4LS (London, UK), English \\
2011 & OWL, as part of ATHENS course (UPM). English \\
2011 & Web Ontology Language (OWL), as part of Artificial Intelligence MSc (UPM). English \\ %(2 hours)
2011 & OWL/Description Logics, as part of the Artificial Intelligence course (UPM). Spanish \\ % (6 hours, 2 groups)
% \item ANECA certificate: Ayudante Doctor.
2005-2008 & OWL tutorials for biologists (University of Manchester, UK). English \\

\end{longtable}


\section*{Ph.D. panels}

\begin{longtable}{p{0.5in}|p{5.5in}}

2016 & Alejandro Rodr\'iguez Iglesias, ``FAIR approaches applied to unraveling plant-pathogen interactions data and RNA processing evolution'', UPM, Spain \\

2013 & Meifania Monica Chen, ``Lipoprotein Ontology: A Formal Representation of Lipoproteins'', Curtin University. Australia \\

2012 & Jose Antonio Mi\~narro-Gim\'enez, ``Entorno para la gesti\'on sem\'antica de informaci\'on biom\'edica en investigaci\'on traslacional''. UM, Spain \\

2011 & Doris Mej\'ia \'Avila, ``Estrategia de interoperabilidad sem\'antica en el contexto de integraci\'on de conocimiento geogr\'afico y ambiental. Caso de aplicaci\'on: Biodiversity Ontology''. UPM, Spain \\

\end{longtable}


\section*{Student supervision}

\begin{longtable}{p{0.5in}|p{5.5in}}

2017 & Denis Mishel Uchuari, internship at Eurohelp, Bachelor's Degree in Computer Engineering in Management and Information Systems. UPV/EHU, Spain \\
2015 &  Salvador Alonso Mart\'inez, ``Imagen Docker para pipelines de Metagen\'omica'', Bioinformatics MSc project. UM, Spain \\


\end{longtable}

% \section*{Technical skills}
% \begin{itemize}
% 	\item Semantic Web and Linked (Open) Data languages (Advanced): RDF, RDFS, SPARQL, OWL, SWRL, SHACL, JSON-LD, RDFa 
% 	\item Semantic Web and Linked (Open) Data tools (Advanced): OWL API, RDFLib, ONTO-PERL, Jena, RDF4J, Virtuoso, Stardog, Blazegraph, Apache Marmotta, GraphDB, Pubby, Prot\'eg\'e and Prot\'eg\'e server, TopBraid composer, CKAN, D2RQ, RML.io, R2RML, Open Refine, Grafter
% 	\item Programming languages (Medium): Java, Python
% 	\item Programming languages (Basic): Clojure, JavaScript, JSP, Bash, Groovy
%   	\item Markup languages (Basic): XML, XSLT, HTML 5, CSS, \LaTeXe, MarkDown
% 	\item UNIX systems (Medium): GNU/Linux (Debian, Ubuntu, CentOS, Red Hat)
% 	\item Software development (Basic): Maven, Ant, Leiningen, Eclipse, Subversion, Mercurial, Git (GitHub, GitLab), Travis CI, Jenkins, CodeCov
% 	\item Project management (Basic): Jira, Trac, GitHub projects
% 	\item Software development methodologies (Basic): Scrum, Kanban
% 	\item SQL and NoSQL Databases (Basic): MySQL, PostgreSQL, Neo4j, MongoDB
% 	\item Statistical analysis (Basic): R
% 	\item Virtualisation (Basic): Docker, Virtual Box
% 	%\item Documentation systems (Basic): Sphinx
% 	%\item File based data storage (Basic): HDF5, YAML, JSON
% 	%\item Web Services (Basic): SADI framework, NCBO services
% 	%\item Web (Basic): Galaxy (Bioinformatics workflows), Apache, Nginx, Tomcat, Jetty, lighttp, Solr/Lucene, Wordpress, Drupal, Jekyll, NodeJS, Spring-MVC
% 	%\item Cloud computing (Basic): Amazon EC2
% \end{itemize}

% \section*{Learning experience}
% \begin{longtable}{p{0.5in}|p{5.5in}}
% 2017 & Predictive modelling. 20 hours course at Tecnalia \\
% 2003 & Microsoft Visual Basic. 20 hours course at UPV/EHU \\
% 2003 & Occupational hazards prevention.  30 hours course at UPV/EHU \\
% 2003 & UNIX. 24 hours course at UPV/EHU \\
% 2002 & ISO 9001:2000. 30 hours course at UPV/EHU \\
% 2002 & XML. 16 hours course at University of Deusto \\
% 1998 & Environmental sciences (Ekoeskola). 65 hours course at Erreka Ecologist Platform \\
% \end{longtable}

% \section*{Other merits}
% \begin{itemize}
% 	\item ANECA certificate: Ayudante Doctor
% 	\item Member of Open Knowledge Foundation Spain (OKFN-ES) and Basque Association of Biologists (COBE)
% 	\item Former member of the W3C Semantic Web Health Care and Life Sciences Interest Group and the Spanish Association of Linked Data (AELID)
% \end{itemize}

% \section*{Useful profiles}
% \begin{itemize}
%   \item LinkedIn: \href{http://es.linkedin.com/in/mikeleganaaranguren}{http://es.linkedin.com/in/mikeleganaaranguren}
% 	\item GitHub: \href{http://github.com/mikel-egana-aranguren}{http://github.com/mikel-egana-aranguren}
% 	\item Google Scholar: \href{http://scholar.google.com/citations?user=JsMMKnoAAAAJ}{http://scholar.google.com/citations?user=JsMMKnoAAAAJ}
% 	\item Scopus: \href{http://www.scopus.com/authid/detail.url?authorId=16038705500}{http://www.scopus.com/authid/detail.url?authorId=16038705500}
% 	\item ResearchGate: \href{http://www.researchgate.net/profile/Mikel_Egana}{http://www.researchgate.net/profile/Mikel\_Egana}
% 	\item ORCID: \href{http://orcid.org/0000-0001-8081-1839}{http://orcid.org/0000-0001-8081-1839}
% 	\item ResearcherID: \href{http://www.researcherid.com/rid/K-6878-2014}{http://www.researcherid.com/rid/K-6878-2014}
% \end{itemize}



% \section*{References}

% Give name and full title.


% Footer
\bigskip
\begin{center}
  \begin{footnotesize}
    Last updated: \today
  \end{footnotesize}
\end{center}



\nobibliography{MyPubs}
\end{document}
